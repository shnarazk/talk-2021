\NeedsTeXFormat{LaTeX2e}
%\PassOptionsToClass{handout}{beamer}
\documentclass{beamer}
\usepackage{beamerPack}
\usepackage[boxed,ruled,vlined]{algorithm2e}
\usepackage{setspace}
\usepackage[05]{../lecture}
\def\x{\hspace{4.1ex}}    %BETWEEN TWO 1-DIGIT NUMBERS
\def\y{\hspace{3.4ex}}  %BETWEEN 1 AND 2 DIGIT NUMBERS
\def\z{\hspace{2.7ex}}    %BETWEEN TWO 2-DIGIT NUMBERS
\subtitle{}
\begin{document}

\begin{frame}[fragile]{}
\titlepage
\end{frame}

\section{dynamic programming}		%%%%%%%%
\subsection{}

\begin{frame}[fragile]{動的計画法}{}
\begin{block}{動的計画法(dynamic programming)}
\begin{itemize}%\itemsep8pt
\item 再帰的に分割された部分問題の(最適)解から元問題の(最適)解を構成
\item 部分問題の重複性の利用
\end{itemize}
\end{block}

\footnotetext{線形計画法は不等式に関する問題}
\end{frame}

\section{algorithm design}		%%%%%%%%
\subsection{}


\begin{frame}[fragile]{重複を避ける}{フィボナッチ数}
\begin{codeof}{language=Rust}{}
fn fib(n: usize) -> usize {
    let mut memo = [0; 10000];
    fn f(n: usize) -> usize {
        if n < 2 { return 1; }
        if memo[n] != 0 { return memo[n]; }
        let result = f(n - 2) + f(n - 1);
        memo[n] = result;
        result
    }
    f(n)
}
\end{codeof}
\end{frame}

\begin{frame}[fragile]{空間計算量を下げる}{フィボナッチ数}
\begin{codeof}{language=Rust}{方向の逆転}
fn fib(n: usize) -> usize {
    fn f(k: usize, memo1: usize, memo2: usize) -> usize{
        if n == k { return memo1 + memo2; } 
        f(k + 1, memo2 + memo1, memo1);
    }
    if n <= 2 { 1 } else { f(3, 1, 1) }
}
\end{codeof}

\begin{codeof}{language=Rust}{どうしても再帰がわからない人向け(実行時間は変わらない)}
fn fib(n: usize) -> usize {
    if n <= 2 { return 1; }
    let (mut memo2, mut memo1) = (0, 0);
    for k in 2..n {
        (memo1, memo2) = (memo2 + memo1, memo1);
    }
    memo1 + memo1
}
\end{codeof}
\end{frame}

\begin{frame}[fragile]{素数判定}{}
\[
\forall k \in [2, n-1] : n \% k \ne 0.
\]

\begin{codeof}{language=Rust}{$O(N)$}
fn is_prime(n: usize) -> bool {
    for i in 2..n-1 {
        if n % i == 0 { return false; } 
    }
    true
}
\end{codeof}
高速化
\begin{enumerate}%\itemsep8pt
\item 偶数は調べない
\item 素数しか調べない(そのためにはn以下の全ての素数をまず調べないといけないのだが)
\end{enumerate}
\end{frame}

\begin{frame}[fragile]{素数判定:重複を省く}{}
\[
\forall k \in [2, n-1] : n \% k \ne 0.
\]

\begin{itemize}%\itemsep8pt
\item 6000が2で割れた。3000でも割れる。
\item Xがkで割れた。X/kでも割れる。
\end{itemize}

\[
\forall k \in [2, \lceil\sqrt{n}\rceil] : n \% k \ne 0.
\]

\end{frame}

\begin{frame}[fragile]{問題の再帰構造の次元を落とす(線形化)}{}
どんな例を用意していたのか思い出せない。
\end{frame}

\begin{frame}[fragile]{例題}{\href{}{advent-of-code 2020}より}
a
\end{frame}

\begin{frame}[fragile]{動的計画法}{}
ここでこの術語を出すのがいいかどうか。
\end{frame}

\begin{frame}[fragile]{メモ化}{\href{https://ja.wikipedia.org/wiki/メモ化}{\beamergotobutton{wikipedia}}}
X型の引数からY型の計算結果を求める関数に対してXからYへのメモを作ることで時間計算量を減らす
\begin{codeof}{language=Rust}{メモ化の一般形}
fn foo (問題指定) -> 解の型 {
  static memo: Hash<問題指定の型, 解の型>;
  // メモの確認
  if memo[問題指定]が有効な値を持つ {
    return memo[問題指定];
  }

   本来の計算:問題指定から解を求める

  // メモする
  memo[問題指定] = 解;
  return 解;
}
\end{codeof}

\footnotetext[1]{定型化されているので言語によってはこのコードを自動生成する}
\end{frame}

\begin{frame}[fragile]{fibに対するメモ化の例}{\href{https://ja.wikipedia.org/wiki/メモ化}{\beamergotobutton{repl it}}}
\begin{codeof}{language=Rust}{fib (memo.rs)}
pub fn fib(n: usize) -> usize {
    if let Ok(hash) = MEMO.read() {
        if let Some(r) = hash.get(&n) {
            return *r;
        }
    }
    let n_1 = if n <= 2 { 1 } else { fib(n - 1) };
    let n_2 = if n <= 2 { 0 } else { fib(n - 2) };
    let result = n_1 + n_2;
    if let Ok(mut hash) = MEMO.write() {
        hash.insert(n, result);
    }
    result
}
\end{codeof}

\begin{spacing}{0.7}\fontsize{6}{6}\selectfont
メモは毎回初期化されないようグローバル(static)変数にしたい。
Rustではグローバル変数への操作は同期操作を使ったスレッド安全なものしか受け付けないので、
ライブラリのインポートや変数定義(ここでは省略)、アクセス(L.2--6, L.10-12)が面倒になっている。
\end{spacing}
\end{frame}

\begin{frame}[fragile]{効果の確認}{}

\begin{codeof}{language=Rust}{}
fn main() {
    let n = 46;
    // mainから1回しか呼ばなくても再帰しているのでメモ化は有効
    println!("fib({:>2}) = {:>11}", n, fib(n));
    println!("fib({:>2}) = {:>11}", n, slow_fib(n));
    // メモに残っているので同じ計算では2回目以降はO(1)
    for i in 30..=48 {
        println!("fib({:>2}) = {:>11}", n, fib(n));
        println!("slw({:>2}) = {:>11}", n, slow_fib(n));
    }
    // 引数が違ってもメモに残っているので新規分だけ計算
    for i in 30..=48 {
        println!("fib({:>2}) = {:>11}", i, fib(i));
        println!("slw({:>2}) = {:>11}", i, slow_fib(i));
    }
}
\end{codeof}
\end{frame}

\end{document}

