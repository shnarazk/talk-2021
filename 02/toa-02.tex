\NeedsTeXFormat{LaTeX2e}
%\PassOptionsToClass{handout}{beamer}
\documentclass{beamer}
\usepackage{beamerPack}
\usepackage[boxed,ruled,vlined]{algorithm2e}
\usepackage[02]{../lecture}
\subtitle{}
\begin{document}

\begin{frame}[fragile]{}
\titlepage
\end{frame}

\section{time \& space complexity}		%%%%%%%%
\subsection{}

\begin{frame}[fragile]{計算時間の定義}{}
ハードウェアや記述言語ではなくアルゴリズムの優劣を評価
\begin{itemize}%\itemsep8pt
\item if文の実行回数
\item 代入文の実行回数 -- 入力データに強く依存
\item 繰り返し文(の終了条件判定)の実行回数 -- この問題では馬鹿にならない
\end{itemize}

\begin{itemize}\itemsep8pt
\item 簡単のため最悪評価で考える
\item データの大きさ(個数)$N$を考慮する
\end{itemize}

\vfill
従って、大きさ$N$のデータに対する比較文の実行回数$X$を比較
\end{frame}

\begin{frame}[fragile]{時間計算量、空間計算量}{}
\begin{block}{時間計算量}
絶対時間ではなく回数を基準として計算の実行に必要な時間
\end{block}
\begin{block}{(追加)空間計算量}
その計算を実行するために必要なメモリ量。ただしデータそのものは含めないことが多い(追加で必要な分だけを対象)。
Byte数よりも個数が単位に使われる。
\end{block}
\end{frame}

\begin{frame}[fragile]{二分探索における時間計算量1}{}

\begin{exampleblock}{線形探索の時間計算量と空間計算量}
\begin{itemize}%\itemsep8pt
\item 繰り返し文の本体を$N$回繰り返し、その度に1回の比較を実行するから時間計算量は$N$
\item データそのもの以外にローカル変数が1個。
\end{itemize}
\end{exampleblock}
\end{frame}

\begin{frame}[fragile]{二分探索における時間計算量:帰納法版}{}

\begin{codeof}{language=Rust}{再帰版bsearch}
fn lsearch<T>(v: &[T], c: ...) -> bool {
   if 長さが0なら { return false; } 
   if 先頭が条件cを満たすなら { return true; }
   bsearch(v[1..], c) // 先頭以外すなわちN-1個を対象に調べる
}
\end{codeof}

大きさ$N-1$のデータに対する時間計算量$T_{N-1}$に1回の比較を追加したものが大きさ$N$の時間計算量$T_{N}$。

\begin{align*}
T_{N} = T_{N-1} + 1
\end{align*}

従って漸化式を解くと(あるいは帰納法を使うと)時間計算量は$N$。
\end{frame}

\begin{frame}[fragile]{二分探索における空間計算量}{}
補足?
\end{frame}

\section{$O$-notation}		%%%%%%%%
\subsection{}

\begin{frame}[fragile]{$O$記法}{}
線形探索の時間計算量が$N$ならより速い探索として$0.9N$や$0.1N^2$などが望まれる?

Nに対する変化がより重要:係数ではなく次数に注目するための一種の正規化

\vfill
実数値関数 f(x) と g(x) に対し、

\begin{align*}
\exists x_{0}, \exists M > 0,  \forall x > x_{0} : |f(x)| < M | g(x) |
\end{align*}
が成立するとき(またその時に限り)、

\begin{align*}
f(x) &= O(g(x))
\end{align*}
\end{frame}

\begin{frame}[fragile]{$O$-記法の解釈}{}

\begin{itemize}%\itemsep8pt
\item $f$は$g$で上から抑えられる -- 理論的上限を表明
\item $x_0$ -- データ数が小さいうちは気にしない
\item $M$が適当に設定できる -- 係数ではなく次数が大事
\end{itemize}

\begin{align*}
1.5 x & = O(x) \\
\sum_{1}^{N} N & = O(N^2) \\
1.5x^2 + 2x & = O(x^2 + x) \\
         & = O(x^2) \\
2.14x^3 & \approx O(x^3)
\end{align*}
最後のは間違い。近似関係ではなく等値関係。

\end{frame}

\begin{frame}[fragile]{$\Omega(-)$, $\Theta(-)$}{}
\begin{align*}
\exists x_{0}, \exists M > 0,  \forall x > x_{0} : |f(x)| > M | g(x) |
\end{align*}
が成立するとき(またその時に限り)、

\begin{align*}
f(x) &= \Omega(g(x))
\end{align*}

\vfill
\begin{align*}
f(x) = O(g(x)) && f(x) = \Omega(g(x))
\end{align*}
が成立するとき(またその時に限り)、

\begin{align*}
f(x) &= \Theta(g(x))
\end{align*}
\end{frame}

\begin{frame}[fragile]{線形探索の計算量}{}

\begin{itemize}%\itemsep8pt
\item 線形探索の時間計算量は$O(N)$である。
\item 線形探索の時間計算量のオーダーは$N$である。
\item 線形探索の空間計算量は$O(1)$である\footnotemark。
\end{itemize}

\footnotetext[1]{添字のためなどにローカル変数をいくつか定義しても$N + 1 = O(N)$。}

\end{frame}

\begin{frame}[fragile]{二分探索の計算量}{}
\scalebox{0.7}{
\begin{algorithm}[H]
\SetKwComment{Comment}{}{}
\BlankLine
i = 0; j = z.len() - 1\Comment*{\sl\small\color[gray]{0.5}検索範囲}
\While{i <= j} {
  \If{ v[(i + j) / 2]が等しい}{
    \Return{発見}
  }
  \eIf{v[(i + j) / 2]が小さい}{
    i = (i + j) / 2 + 1\Comment*{\sl\small\color[gray]{0.5}検索範囲の左半分を破棄}
  }{
    j = (i + j) / 2 - 1\Comment*{\sl\small\color[gray]{0.5}検索範囲の右半分を破棄}
  }
}
\Return{失敗}\;
\caption[page]{二分探索}
\end{algorithm}
}

\vfill

\begin{itemize}%\itemsep8pt
\item データの大きさが$N$の場合の繰り返し回数\footnotemark: $\log_{2}(N)$
\item 各繰り返しの中で実行する条件比較は2回=$O(1)$回
\item ローカル変数が二つ
\end{itemize}
\end{frame}

\begin{frame}[fragile]{代表的探索の計算量}{}

{%\fontsize{9}{10}\selectfont
\begin{tabular}[h]{|c|r|r|r|}
\CH アルゴリズム & (最悪)時間計算量 & 最良時間-- & 空間-- \\
\CL 線形探索 & $O(N)$ & $O(1)$ & $O(1)$ \\
\CL 二分探索 & $O(\log(N))$ & $O(1)$ & $O(1)$ \\
\end{tabular}
}
\end{frame}

\begin{frame}[fragile]{より速い探索}{}

一般には無理。

\begin{block}{並列探索}
データをパーティショニング。固定のM台使用するなら$O(N/M) = O(N)$。
データの大きさに合わせて増加させるなら$O(N)$以下が可能。
\end{block}

\begin{block}{ハッシュ表とハッシュ関数}
\begin{itemize}%\itemsep8pt
\item 時間計算量は$O(1)$
\item 空間計算量は$O(データの潜在的総数)$
\end{itemize}
\end{block}

二分探索は速いが、計算量が大きな前処理が必要。なんとかならないか:赤黒木などの〇〇木
\end{frame}

\end{document}
