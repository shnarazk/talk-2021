\NeedsTeXFormat{LaTeX2e}
%\PassOptionsToClass{handout}{beamer}
\documentclass{beamer}
\usepackage{beamerPack}
\usepackage[boxed,ruled,vlined]{algorithm2e}
\usepackage[02]{../lecture}
\subtitle{}
\begin{document}

\begin{frame}[fragile]{}
\titlepage
\end{frame}

\section{time \& space complexity}		%%%%%%%%
\subsection{}

\begin{frame}[fragile]{計算時間の定義}{}
if文の回数
\end{frame}

\begin{frame}[fragile]{計算時間の定義}{}
最悪評価
\end{frame}

\begin{frame}[fragile]{計算時間の定義}{}
データの大きさ(個数)N
\end{frame}

\begin{frame}[fragile]{計算時間の定義}{}
長さNの配列から最悪
\end{frame}

\begin{frame}[fragile]{時間計算量、空間計算量1}{}
長さNの配列から最悪
\end{frame}

\begin{frame}[fragile]{時間計算量、空間計算量2}{}
N個のデータから検索する状況においてどれほどのメモリが必要か
\end{frame}

\begin{frame}[fragile]{線形探索における時間計算量1}{}
\end{frame}

\begin{frame}[fragile]{線形探索における空間計算量2}{}
\end{frame}


\begin{frame}[fragile]{二分探索における時間計算量2}{}
\end{frame}

\begin{frame}[fragile]{二分探索における空間計算量4}{}
\end{frame}

\begin{frame}[fragile]{二分探索における空間計算量5}{}
\end{frame}

\section{$O$-notation}		%%%%%%%%
\subsection{}

\begin{frame}[fragile]{big-$O$ notation}{}
a
\end{frame}

\begin{frame}[fragile]{$O$-記法の解釈}{}
a
\end{frame}

\begin{frame}[fragile]{$\Sigma(-)$, $\Omega(-)$}{}
a
\end{frame}

\begin{frame}[fragile]{検索アルゴリズムの計算量のオーダー}{}
a
\end{frame}

\begin{frame}[fragile]{検索アルゴリズムの計算量のオーダー}{}
a
\end{frame}

\end{document}
